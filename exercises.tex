\documentclass{book}

\usepackage[T1,T2A]{fontenc}
\usepackage[utf8]{inputenc}
\usepackage[english,russian]{babel}

\usepackage{graphicx}
\usepackage{amsmath}

\title{Решение упражнений из книги Искусство программирования на компьютере Д. Кнута \thanks{}}
\author{Dim Ch}
\date{July 2022}

\graphicspath{ {images/} }

\begin{document}

\chapter{Основные понятия}

\section{Алгоритмы}
\subsection*{}
\subsubsection{1.}

Необходимо преобразовать четверку $ (a, b, c, d) \textrm{ в } (b, c, d, a)$.

\begin{flalign*}
  t \leftarrow a, a \leftarrow b, b \leftarrow c, c \leftarrow d, d \leftarrow t && \\
\end{flalign*}

\subsubsection{2.}

Докажте, что в начале выполнения шага E1 $m$ всегда больше $n$, за исключением только первого случая выполнения этого шага.

Рассмотрим второй и последующий случаи выполнения шага E1. В этом случае шагу E1 всегда предшествует шаг E3, на котором $m$ принимается равным $n$, а $n$ -- $r$. Шагу E3 всегда предшествует шаг E1, на котором $r$ вычисляется как остаток от деления $m$ на $n$ при чем $(0 \leq r < n)$. Итого имеем

\begin{flalign*}
  \textrm{E1. } & 0 \leq r < n && \\
  \textrm{E3. } & m \leftarrow n, n \leftarrow r \Rightarrow 0 \leq n < m && \\
  \textrm{E1. } & 0 \leq n < m && \\
\end{flalign*}

\subsubsection{3.}

Изменить алгоритм E (из соображений эффективности) таким образом, чтобы исключить из него все тривиальные операции замены типа "$m \leftarrow n$".

\textbf{Алгоритм F} (Алгоритм Евклида без тривиальных присваиваний). Даны два целых положительных числа $m$ и $n$. Требуется найти их НОД.

\textbf{E1}. [\textit{Остаток от деления 1}] Вычислить остаток от деления $m$ на $n$ и сохранить его в $m$.

\textbf{E2}. [\textit{Сравнение с 0}] Если $m$ равно 0, ответ $n$.

\textbf{E3}. [\textit{Остаток от деления 1}] Вычислить остаток от деления $n$ на $m$ и сохранить его в $n$.

\textbf{E4}. [\textit{Сравнение с 0}] Если $n$ равно 0, ответ $m$.

\textbf{E5}. [\textit{Цикл}] Перейти на шаг 1.

\subsubsection{4.}

НОД 2166 и 6099 равен 57.

\subsubsection{5.}

Метод вычисления считается алгоритмом, если он удовлетворяет следующим условиям:
\begin{enumerate}
\item Конечность. Алгоритм -- это конечный метод вычислений.
\item Определенность. Программа -- метод вычислений на формальном языке.
\item Ввод.
\item Вывод.
\item Эффективность.
\end{enumerate}

Для процедуры чтения этой книги не выполняются условия:
\begin{itemize}
\item эффективности (например, не существует простого способа выполнить все упражнения, приведенные в книге);
\item конечности;
\item наличия определенного вывода (хотя неформально можно описать ожидаемые результаты от прочтения клиги).
\end{itemize}

Остальные условия условно можно считать выполненными:
\begin{itemize}
\item наличие ввода (текст книги);
\item определенность (хотя описание процедуры приведено на неформальном языке, шаги являются достаточно понятными и недвусмысленными).
\end{itemize}

В описании процедуры нет названия алгоритма, краткого описания шагов и завершающей вертикальной черты.

\subsubsection{6.}

Чему равно $T_5$ (среднее число случаев выполнения шага E1 при $n=5$)?

По условию алгоритма E $m$ -- положительное целое число. Рассмотрим все $ m \leq n$:

\begin{center}
\begin{tabular}{ c c }
 $m$ & $T_5$  \\ 
 1 & 2  \\  
 2 & 3  \\
 3 & 4  \\
 4 & 3  \\
 5 & 1  \\
\end{tabular}
\end{center}

Если $m < n$, то на первой итерации значения $m$ и $n$ меняются местами, а затем происходит выполнение алгоритма для $m=5$ и $n < m$.
Если $m > n$, то на первой итерации происходит вычисление остатка от деления $m$  на $n$, а затем алгоритм также выполняется для $m=5$ и $n < m$.
Если $m \bmod n = 0$ (в т.ч. при $m=n$), то алгоритм завершается на первой же итерации.

Таким образом, при увеличении $m$ значение $T_5$ будет последовательно принимать значения из таблицы выше. Среднее значение $T_5 = (2+3+4+3+1)/5 = 2.6$.

\subsubsection{7.}

Пусть $m$ известно, а $n$ -- любое целое положительное число. Пусть $U_m$ -- среднее число случаев выполнения шага E1 из алгоритма E. Покажите, что $U_m$ четко определено. Существует ли какая либо связь между $U_m$ и $T_m$?

Рассмотрим все случаи, когда $n>m$. На первом шаге значения $m$ и $n$ меняются местами. Среднее число дальнейших итераций равно $T_m$.
Т.е. среднее число $U_m$ в этом случае равно $T_m + 1$.

Если $n<m$, то $U_m = T_m - 1$, т.к. в это же состояние можно было прийти, обменяв местами $m$ и $n$ на первой итерации, как это показано в предыдущем упражнении.

Для $m=n$ $U_m = T_m$.

\section{Математическое введение}
\subsection{Математическая индукция}
\subsection{Числа, степени и логарифмы}
\subsection{Суммы и произведения}
\subsection{Целочисленные функции и элементарная теория чисел }
\subsection{Перестановки и факториалы}

\subsection{Биномиальные коэффициенты}

\subsubsection{10.}
Пусть $p$ --- простое число. Покажите:

\paragraph{e.}

\begin{flalign} \label{eq:1_2_6__10_e_1}
  \binom{n}{k} \equiv \binom{\lfloor n/p \rfloor}{\lfloor k/p \rfloor} \binom{n \bmod p}{k \bmod p} (\textrm{по модулю} \ \ p) &&
\end{flalign}

Основная формула биномиального коэффициента:

\begin{flalign*}
  \binom{n}{k} = \frac {n (n-1) (n-2) \dots (n-k+1)} {k (k-1) (k-2) \dots 1} &&
\end{flalign*}

Рассмотрим $ k \bmod p $ первых сомножителей в знаменателе. Все они делятся на $p$ с остатком. 
Если $ k \bmod p > 0 $, то

\begin{flalign*}
  \prod_{i=0}^{(k \bmod p)-1} k-i & \equiv \prod_{i=1}^{k \bmod p} i \ \ & (\textrm{по модулю } p) \\
  k & \equiv k \bmod p \ \ & (\textrm{по модулю } p) \\
  k - 1 & \equiv (k \bmod p) - 1 \ \ & (\textrm{по модулю } p) \\
  \dots & \equiv \dots & \\
  k - (k \bmod p) + 2 & \equiv 2  \ \ & (\textrm{по модулю } p) \\
  k - (k \bmod p) + 1 & \equiv 1  \ \ & (\textrm{по модулю } p) \\
\end{flalign*}

Например, для $ k=17, p=7 $:

\begin{flalign*}
  17 \bmod 7 & = 3 & \\
  (17 \bmod 7) - 1 & = 2 & \\
  \prod_{i=0}^{2} 17-i  & \equiv \prod_{i=1}^{3} i \ \ & (\textrm{по модулю } 7) \\
  17 \cdot 16 \cdot 15  & \equiv 3 \cdot 2 \cdot 1 \ \ & (\textrm{по модулю } 7) \\
  4080 & \equiv 6  \ \ & (\textrm{по модулю } 7) \\
  4080 \bmod 7 & = 6
\end{flalign*}

Эта формула также справедлива для случая, когда $k$ делится на $p$ без остатка. Например, для $ k=14, p=7 $:

\begin{flalign*}
  14 \bmod 7 & = 0 & \\
  (14 \bmod 7) - 1 & = -1 & \\
  \prod_{i=0}^{-1} 14-i  & \equiv \prod_{i=1}^{0} i \ \ & (\textrm{по модулю } 7) \\
  1  & \equiv 1 \ \ & (\textrm{по модулю } 7)
\end{flalign*}

Теперь рассмотрим $ k \bmod p $ первых сомножителей в числителе $ n (n - 1) \dots (n - (k \bmod p) + 1) $.
Среди них может не оказаться сомножителя кратного $p$. Тогда:

\begin{flalign*}
  \prod_{i=0}^{(k \bmod p)-1} n-i & \equiv (n \bmod p) ((n \bmod p) - 1) \dots ((n \bmod p) - (k \bmod p) + 1) \ \ & (\textrm{по модулю } p) \\
  n & \equiv n \bmod p \ \ & (\textrm{по модулю } p) \\
  n - 1 & \equiv (n \bmod p) - 1 \ \ & (\textrm{по модулю } p) \\
  \dots & \equiv \dots & \\
  n - (k \bmod p) + 2 & \equiv (n \bmod p) - (k \bmod p) + 2  \ \ & (\textrm{по модулю } p) \\
  n - (k \bmod p) + 1 & \equiv (n \bmod p) - (k \bmod p) + 1  \ \ & (\textrm{по модулю } p) \\
\end{flalign*}

Например, для $ n=20, k=17, p=7 $:

\begin{flalign*}
  20 \bmod 7 & = 6 & \\
  (20 \bmod 7) - 1 & = 5 & \\
  \prod_{i=0}^{2} 20-i  & \equiv 6 \cdot 5 \cdot 4 \ \ & (\textrm{по модулю } 7) \\
  20 \cdot 19 \cdot 18  & \equiv 6 \cdot 5 \cdot 4 \ \ & (\textrm{по модулю } 7) \\
  6840 & \equiv 120  \ \ & (\textrm{по модулю } 7) \\
  6840 \bmod 7 & = 1 &\\
  120 \bmod 7 & = 1 &
\end{flalign*}

Для случая $ k \bmod p = 0 $ произведение в числителе, также как и в знаменателе, обращается в $ 1 $.
Если среди первых $ k \bmod p $ сомножителей в числителе встретится число, кратное $p$, то

\begin{flalign*}
  \prod_{i=0}^{(k \bmod p)-1} n-i & \equiv 0 \ \ & (\textrm{по модулю } p) \\
\end{flalign*}

Подставим получившиеся соотношения в дробь

\begin{flalign*}
  \frac{\prod_{i=0}^{(k \bmod p)-1} n-i}{\prod_{i=0}^{(k \bmod p)-1} k-i} & \equiv \frac{(n \bmod p) ((n \bmod p) - 1) \dots ((n \bmod p) - (k \bmod p) + 1)}{\prod_{i=1}^{k \bmod p} i}  \ \ & (\textrm{по модулю } p) \\
  \frac{\prod_{i=0}^{(k \bmod p)-1} n-i}{\prod_{i=0}^{(k \bmod p)-1} k-i} & \equiv \binom{n \bmod p}{k \bmod p} \ \ & (\textrm{по модулю } p)
\end{flalign*}

Для случая $ \prod_{i=0}^{(k \bmod p)-1} n-i \equiv 0 \ \ (\textrm{по модулю } p) $ соотношения также справедливы, т.к. $ \binom{0}{k} = \Bigl\{ \genfrac{}{}{0pt}{}{0 \textrm{ при } k > 0}{1 \textrm{ при } k = 0} $.

Рассмотрим далее оставшиеся сомножители в знаменателе: $ \prod_{i=1}^{k-(k \bmod p)} i $. Например, для $ k=17, p = 7 $ имеем $ \prod_{i=1}^{14} i $. Можно выделить $ \lfloor k / p \rfloor $ групп по $ p $ элементов в каждой. При этом в каждой группе существует только один элемент, кратный $ p $, а остальные эквиваленты $ (p-1)! $. То же самое утверждение справедливо и для числителя.

Тогда элементы некратные $p$ сокращаются в числителе и знаменателе. Рассмотрим $ \lfloor k / p \rfloor $ элементов кратных $p$. Разделим каждый сомножитель в числителе и знаменателе на $ p $. Получим в знаменателе $ \lfloor k / p \rfloor ! $, т.к. первое число равно $ \lfloor k / p \rfloor $, а каждое последующее должно уменьшаться на $1$. В числителе получим $ \lfloor k / p \rfloor $ сомножителей, первым из которых является $ \lfloor (n - (k \bmod p)) / p \rfloor $, а последующие также уменьшаются на единицу. Т.е. получим биномиальный коэффициент

\begin{flalign*}
  \binom{\lfloor (n - (k \bmod p)) / p \rfloor}{\lfloor k / p \rfloor}
\end{flalign*}

При этом

\begin{flalign*}
  \lfloor (n - (k \bmod p)) / p \rfloor = \biggl\{ \genfrac{}{}{0pt}{0}
          {\lfloor n / p \rfloor \textrm{, если } k \bmod p \leq n \bmod p}
          {\lfloor n / p \rfloor - 1 \textrm{, если } k \bmod p > n \bmod p}
\end{flalign*}

Учтем, что при $ k \bmod p > n \bmod p $ второй сомножитель $ \binom{n \bmod p}{k \bmod p} = 0 $, поэтому значение первого сомножителя во всех случаях можно оставить равным $ \lfloor n / p \rfloor $.

Интересным следствием доказанной формулы является тот факт, что

\begin{flalign} \label{eq:1_2_6__10_e_2}
  \binom{n}{k} \equiv 0 \ \ (\textrm{по модулю } p) \textrm{, если } k \bmod p > n \bmod p 
\end{flalign}

\paragraph{a.}

\begin{flalign} \label{eq:1_2_6__10_a_1}
  \binom{n}{p} \equiv \biggl \lfloor \frac{n}{p} \biggl \rfloor \ \ (\textrm{по модулю} \ \ p) &&
\end{flalign}


Подставим в формулу \ref{eq:1_2_6__10_e_1} $p$ вместо $k$. Получим

\begin{flalign*}
  \binom{n}{p} & \equiv \binom{\lfloor n/p \rfloor}{\lfloor p/p \rfloor} \binom{n \bmod p}{p \bmod p} & (\textrm{по модулю} \ \ p) && \\
  \binom{n}{p} & \equiv \binom{\lfloor n/p \rfloor}{1} \binom{n \bmod p}{0} & (\textrm{по модулю} \ \ p) && \\
  \binom{n}{p} & \equiv \lfloor n/p \rfloor \cdot 1 & (\textrm{по модулю} \ \ p) && \\
  \binom{n}{p} & \equiv \lfloor n/p \rfloor & (\textrm{по модулю} \ \ p) && \\
\end{flalign*}

\paragraph{b.}

\begin{flalign} \label{eq:1_2_6__10_b_1}
  \binom{p}{k} \equiv 0 \ \ (\textrm{по модулю} \ \ p) \textrm{ для } 1 \leq k \leq p-1 &&
\end{flalign}

Подставим в формулу \ref{eq:1_2_6__10_e_1} $p$ вместо $n$. Получим

\begin{flalign*}
  \binom{p}{k} & \equiv \binom{\lfloor p/p \rfloor}{\lfloor k/p \rfloor} \binom{p \bmod p}{k \bmod p} & (\textrm{по модулю} \ \ p) && \\
  \binom{p}{k} & \equiv \binom{1}{\lfloor k/p \rfloor} \binom{0}{k \bmod p} & (\textrm{по модулю} \ \ p) && \\
  & k \bmod p = k \textrm{ для } 1 \leq k \leq p-1 & && \\
  & \lfloor k/p \rfloor = 0 \textrm{ для } 1 \leq k \leq p-1 & && \\
  \binom{p}{k} & \equiv \binom{1}{0} \binom{0}{k} & (\textrm{по модулю} \ \ p) && \\
  \binom{p}{k} & \equiv 1 \cdot 0 & (\textrm{по модулю} \ \ p) && \\
  \binom{p}{k} & \equiv 0 & (\textrm{по модулю} \ \ p) && \\
\end{flalign*}

\paragraph{c.}

\begin{flalign*}
  \binom{p-1}{k} \equiv (-1)^{k} \ \ (\textrm{по модулю} \ \ p) \textrm{ для } 0 \leq k \leq p-1 &&
\end{flalign*}

Если $k=0$, а $p>0$, то $ \binom{p-1}{k} = \binom{p-1}{0} = 1 $ и $ (-1)^k = 1 $.

Подстановка $p-1$ вместо $n$ в формулу \ref{eq:1_2_6__10_e_1} ничего не дает, т.к. первый сомножитель обращается в $1$, а второй в исходное выражение:

\begin{flalign*}
  \binom{p-1}{k} & \equiv \binom{\lfloor (p-1)/p \rfloor}{\lfloor k/p \rfloor} \binom{p-1 \bmod p}{k \bmod p} & (\textrm{по модулю} \ \ p) && \\
  \binom{p-1}{k} & \equiv \binom{0}{\lfloor k/p \rfloor} \binom{p-1}{k \bmod p} & (\textrm{по модулю} \ \ p) && \\
  & k \bmod p = k \textrm{ для } 0 \leq k \leq p-1 & && \\
  & \lfloor k/p \rfloor = 0 \textrm{ для } 0 \leq k \leq p-1 & && \\
  \binom{p-1}{k} & \equiv \binom{0}{0} \binom{p-1}{k} & (\textrm{по модулю} \ \ p) && \\
\end{flalign*}

Рассмотрим формулу сложения:

\begin{flalign*}
  \binom{p}{k} & = \binom{p-1}{k} + \binom{p-1}{k-1} && \\
  \binom{p-1}{k} & = \binom{p}{k} - \binom{p-1}{k-1} && \\
\end{flalign*}

Т.к. по формуле \ref{eq:1_2_6__10_b_1} $\binom{p}{k} \equiv 0 \ \ (\textrm{по модулю } p)$, то имеем

\begin{flalign*}
  \binom{p-1}{k} & \equiv - \binom{p-1}{k-1} & (\textrm{по модулю } p) && \\
  \binom{p-1}{0} & \equiv 1 & (\textrm{по модулю } p) && \\
  \binom{p-1}{1} & \equiv -1 = p-1 & (\textrm{по модулю } p) && \\
  \binom{p-1}{2} & \equiv 1 & (\textrm{по модулю } p) && \\
  \dots & \equiv \dots & (\textrm{по модулю } p) && \\
  \binom{p-1}{p-1} & \equiv 1 & (\textrm{по модулю } p) && \\
  \binom{p-1}{k} & \equiv (-1)^{k} & (\textrm{по модулю } p) && \\
\end{flalign*}

Т.к. $p$ -- простое число, а следовательно и нечетное, то $p-1$ -- число четное и $(-1)^{p-1} = 1$. 

\paragraph{d.}

\begin{flalign*}
  \binom{p+1}{k} \equiv 0 \ \ (\textrm{по модулю} \ \ p) \textrm{ для } 2 \leq k \leq p-1 &&
\end{flalign*}

По формуле сложения:

\begin{flalign*}
  \binom{r}{k} & = \binom{r-1}{k} + \binom{r-1}{k-1} && \\
  \binom{p+1}{k} & = \binom{p}{k} + \binom{p}{k-1} && \\
\end{flalign*}

По формуле \ref{eq:1_2_6__10_b_1} $ \binom{p}{k} \equiv 0 \ \ (\textrm{по модулю } p) \textrm{ для } k \geq 1 $, но тогда и $ \binom{p}{k-1} \equiv 0 \ \ (\textrm{по модулю } p) $, т.к. по условию $ k \geq 2 $.

\paragraph{f.}

Если в системе счисления с основанием $p$ числа $n$ и $k$ представляются в виде

\begin{flalign*}
  \genfrac{}{}{0pt}{}{n = a_r p^r + \dots + a_1 p + a_0,}{k = b_r p^r + \dots + b_1 p + b_0,} \textrm{ то }
  \binom{n}{k} \equiv \binom{a_r}{b_r} \dots \binom{a_1}{b_1} \binom{a_0}{b_0} \ \ (\textrm{по модулю } p)
\end{flalign*}

Заметим, что

\begin{flalign*}
  \lfloor n/p \rfloor & = a_r p^{r-1} + \dots + a_2 p + a_1 &&  \\
  n \bmod p           & = a_0 &&
\end{flalign*}

Таким образом по формуле \ref{eq:1_2_6__10_e_1} получаем

\begin{flalign*}
  \binom{n}{k} & \equiv \binom{a_r p^{r-1} + \dots + a_2 p + a_1}{b_r p^{r-1} + \dots + b_2 p + b_1} \binom{a_0}{b_0} \ \ & (\textrm{по модулю } p) && \\
  \binom{n}{k} & \equiv \binom{a_r p^{r-2} + \dots + a_3 p + a_2}{b_r p^{r-2} + \dots + b_3 p + b_2} \binom{a_1}{b_1} \binom{a_0}{b_0} \ \ & (\textrm{по модулю } p) && \\
  \dots & \equiv \dots \ \ & (\textrm{по модулю } p) && \\
  \binom{n}{k} & \equiv \binom{a_r p + a_{r-1}}{b_r p + b_{r-1}} \binom{a_{r-2}}{b_{r-2}} \dots \binom{a_1}{b_1} \binom{a_0}{b_0} \ \ & (\textrm{по модулю } p) && \\
  \binom{n}{k} & \equiv \binom{a_r}{b_r} \binom{a_{r-1}}{b_{r-1}} \binom{a_{r-2}}{b_{r-2}} \dots \binom{a_1}{b_1} \binom{a_0}{b_0} \ \ & (\textrm{по модулю } p) && \\
\end{flalign*}


\end{document}
