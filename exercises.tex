\documentclass{book}

\usepackage[T1,T2A]{fontenc}
\usepackage[utf8]{inputenc}
\usepackage[english,russian]{babel}

\usepackage{graphicx}
\usepackage{amsmath}

\title{Решение упражнений из книги Искусство программирования на компьютере Д. Кнута \thanks{}}
\author{Dim Ch}
\date{July 2022}

\graphicspath{ {images/} }

\begin{document}

\chapter{Основные понятия}

\section{Алгоритмы}
\section{Математическое введение}
\subsection{Математическая индукция}
\subsection{Числа, степени и логарифмы}
\subsection{Суммы и произведения}
\subsection{Целочисленные функции и элементарная теория чисел }
\subsection{Перестановки и факториалы}

\subsection{Биномиальные коэффициенты}

\subsubsection{10.}
Пусть $p$ --- простое число. Покажите:

\paragraph{e.}

\begin{flalign*}
  \binom{n}{k} \equiv \binom{\lfloor n/p \rfloor}{\lfloor k/p \rfloor} \binom{n \bmod p}{k \bmod p} (\textrm{по модулю} \ \ p) &&
\end{flalign*}

Основная формула биномиального коэффициента:

\begin{flalign*}
  \binom{n}{k} = \frac {n (n-1) (n-2) \dots (n-k+1)} {k (k-1) (k-2) \dots 1} &&
\end{flalign*}

Рассмотрим $ k \bmod p $ первых сомножителей в знаменателе. Все они делятся на $p$ с остатком. 
Если $ k \bmod p > 0 $, то

\begin{flalign*}
  \prod_{i=0}^{(k \bmod p)-1} k-i & \equiv \prod_{i=1}^{k \bmod p} i \ \ & (\textrm{по модулю } p) \\
  k & \equiv k \bmod p \ \ & (\textrm{по модулю } p) \\
  k - 1 & \equiv (k \bmod p) - 1 \ \ & (\textrm{по модулю } p) \\
  \dots & \equiv \dots & \\
  k - (k \bmod p) + 2 & \equiv 2  \ \ & (\textrm{по модулю } p) \\
  k - (k \bmod p) + 1 & \equiv 1  \ \ & (\textrm{по модулю } p) \\
\end{flalign*}

Например, для $ k=17, p=7 $:

\begin{flalign*}
  17 \bmod 7 & = 3 & \\
  (17 \bmod 7) - 1 & = 2 & \\
  \prod_{i=0}^{2} 17-i  & \equiv \prod_{i=1}^{3} i \ \ & (\textrm{по модулю } 7) \\
  17 \cdot 16 \cdot 15  & \equiv 3 \cdot 2 \cdot 1 \ \ & (\textrm{по модулю } 7) \\
  4080 & \equiv 6  \ \ & (\textrm{по модулю } 7) \\
  4080 \bmod 7 & = 6
\end{flalign*}

Эта формула также справедлива для случая, когда $k$ делится на $p$ без остатка. Например, для $ k=14, p=7 $:

\begin{flalign*}
  14 \bmod 7 & = 0 & \\
  (14 \bmod 7) - 1 & = -1 & \\
  \prod_{i=0}^{-1} 14-i  & \equiv \prod_{i=1}^{0} i \ \ & (\textrm{по модулю } 7) \\
  1  & \equiv 1 \ \ & (\textrm{по модулю } 7)
\end{flalign*}

Теперь рассмотрим $ k \bmod p $ первых сомножителей в числителе $ n (n - 1) \dots (n - (k \bmod p) + 1) $.
Среди них может не оказаться сомножителя кратного $p$. Тогда:

\begin{flalign*}
  \prod_{i=0}^{(k \bmod p)-1} n-i & \equiv (n \bmod p) ((n \bmod p) - 1) \dots ((n \bmod p) - (k \bmod p) + 1) \ \ & (\textrm{по модулю } p) \\
  n & \equiv n \bmod p \ \ & (\textrm{по модулю } p) \\
  n - 1 & \equiv (n \bmod p) - 1 \ \ & (\textrm{по модулю } p) \\
  \dots & \equiv \dots & \\
  n - (k \bmod p) + 2 & \equiv (n \bmod p) - (k \bmod p) + 2  \ \ & (\textrm{по модулю } p) \\
  n - (k \bmod p) + 1 & \equiv (n \bmod p) - (k \bmod p) + 1  \ \ & (\textrm{по модулю } p) \\
\end{flalign*}

Например, для $ n=20, k=17, p=7 $:

\begin{flalign*}
  20 \bmod 7 & = 6 & \\
  (20 \bmod 7) - 1 & = 5 & \\
  \prod_{i=0}^{2} 20-i  & \equiv 6 \cdot 5 \cdot 4 \ \ & (\textrm{по модулю } 7) \\
  20 \cdot 19 \cdot 18  & \equiv 6 \cdot 5 \cdot 4 \ \ & (\textrm{по модулю } 7) \\
  6840 & \equiv 120  \ \ & (\textrm{по модулю } 7) \\
  6840 \bmod 7 & = 1
  120 \bmod 7 & = 1
\end{flalign*}

Для случая $ k \bmod p = 0 $ произведение в числителе, также как и в знаменателе, обращается в $ 1 $.
Если среди первых $ k \bmod p $ сомножителей в числителе встретится число, кратное $p$, то

\begin{flalign*}
  \prod_{i=0}^{(k \bmod p)-1} n-i & \equiv 0 \ \ & (\textrm{по модулю } p) \\
\end{flalign*}

Подставим получившиеся соотношения в дробь

\begin{flalign*}
  \frac{\prod_{i=0}^{(k \bmod p)-1} n-i}{\prod_{i=0}^{(k \bmod p)-1} k-i} & \equiv \frac{(n \bmod p) ((n \bmod p) - 1) \dots ((n \bmod p) - (k \bmod p) + 1)}{\prod_{i=1}^{k \bmod p} i}  \ \ & (\textrm{по модулю } p) \\
  \frac{\prod_{i=0}^{(k \bmod p)-1} n-i}{\prod_{i=0}^{(k \bmod p)-1} k-i} & \equiv \binom{n \bmod p}{k \bmod p} \ \ & (\textrm{по модулю } p)
\end{flalign*}

Для случая $ \prod_{i=0}^{(k \bmod p)-1} n-i \equiv 0 \ \ (\textrm{по модулю } p) $ соотношения также справедливы, т.к. $ \binom{0}{k} = \Bigl\{ \genfrac{}{}{0pt}{}{0 \textrm{ при } k > 0}{1 \textrm{ при } k = 0} $.

\end{document}
